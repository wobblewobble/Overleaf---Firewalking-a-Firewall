%%%%%%%%%%%%%%%%%%%%%%%%%%%%%%%%%%%%%%%%%%%%%%%%%%%%%%%%%%%%%%%%%%%%%
% LaTeX Template: Project Titlepage Modified (v 0.1) by rcx
%
% Original Source: http://www.howtotex.com
% Date: February 2014
% 
% This is a title page template which be used for articles & reports.
% 
% This is the modified version of the original Latex template from
% aforementioned website.
% 
%%%%%%%%%%%%%%%%%%%%%%%%%%%%%%%%%%%%%%%%%%%%%%%%%%%%%%%%%%%%%%%%%%%%%%

\documentclass[12pt]{report}
\usepackage[a4paper]{geometry}
\usepackage[myheadings]{fullpage}
\usepackage{fancyhdr}
\usepackage{lastpage}
\usepackage{graphicx, wrapfig, subcaption, setspace, booktabs}
\usepackage[T1]{fontenc}
\usepackage[font=small, labelfont=bf]{caption}
\usepackage{fourier}
\usepackage[protrusion=true, expansion=true]{microtype}
\usepackage[english]{babel}
\usepackage{sectsty}
\usepackage{url, lipsum}


\newcommand{\HRule}[1]{\rule{\linewidth}{#1}}
\onehalfspacing
\setcounter{tocdepth}{5}
\setcounter{secnumdepth}{5}

%-------------------------------------------------------------------------------
% HEADER & FOOTER
%-------------------------------------------------------------------------------
\pagestyle{fancy}
\fancyhf{}
\setlength\headheight{15pt}
\fancyhead[L]{Student ID: B00107618}
\fancyhead[R]{Blanchardstown Institute of Technology}
\fancyfoot[R]{Page \thepage\ of \pageref{LastPage}}
%-------------------------------------------------------------------------------
% TITLE PAGE
%-------------------------------------------------------------------------------

\begin{document}

\title{ \normalsize \textsc{Research Skills and Ethics MACS H6021}
		\\ [2.0cm]
		\HRule{0.5pt} \\
		\LARGE \textbf{\uppercase{Firewalk a Firewall On-Premise and in the Cloud}}
		\HRule{2pt} \\ [0.5cm]
		\normalsize \today \vspace*{5\baselineskip}}

\date{}

\author{
		Student ID: B00107618 \\ 
		Blanchardstown Institute of Technology \\
		School of Informatics and Engineering }

\maketitle
\tableofcontents
\newpage

%-------------------------------------------------------------------------------
% Section title formatting
\sectionfont{\scshape}
%-------------------------------------------------------------------------------

%-------------------------------------------------------------------------------
% BODY
%-------------------------------------------------------------------------------

\section*{Introduction}
\addcontentsline{toc}{section}{Introduction}
You will need to develop a series of tests or attacks to prove/disprove that cloud firewalls/security is
worse/better than on-premises firewall/security. The tests will need to be identical for both scenarios. 
\newline
They will have to be your own VMs.
\newline

%-------------------------------------------------------------------------------
% Identify a particular problem
%-------------------------------------------------------------------------------
\newpage
\section*{Identify a problem domain}
\addcontentsline{toc}{section}{Identify a problem domain}
Identify a problem domain
\newline
Security on premise versus security in the cloud, is there a difference?
\newline

\begin{itemize}
\item On-Premise firewall is more secure that a cloud deployed solution.
\item AWS uses Xen and ESX, Azure uses Hyper-V.
\end{itemize}
Fred
\begin{itemize}
\item \lipsum[2]
\end{itemize}

%-------------------------------------------------------------------------------
% Identify a particular problem
%-------------------------------------------------------------------------------
\newpage
\section*{Identify a particular problem}
\addcontentsline{toc}{section}{Identify a particular problem}

Identify a particular problem
\newline
Potential that you could firewalk  (option 2) a cloud firewall easier than an on-premise firewall.
pfSense and VNS3 Available from AWS and Azure

\maketitle
% \begin{tabular}{|l|l|l|} Formats the Left, Centre or Right l,c,r
\begin{tabular}{|l|l|l|}
\hline
\textbf{Firewall} & \textbf{Type} & \textbf{Location}\\
\hline
pfSense & Physical On-Premise & Low end PC\\
\hline
pfSense & Virtual On-Premise & ESXi 6.5\\
\hline
pfSense & Virtual On-Premise & Hyper-V Server 2016\\
\hline
pfSense & Virtual & AWS	Free Tier Machine\\
\hline
pfSense & Virtual Azure & A1 Basic or equivalent to AWS Free Tier\\
\hline
VNS3 & Physical On-Premise & Low end PC* May not be possible\\
\hline
VNS3 & Virtual On-Premise & ESXi 6.5*\\
\hline
VNS3 & Virtual On-Premise & Hyper-V Server 2016*\\
\hline
VNS3 & Virtual AWS & Free Tier Machine\\
\hline
VNS3 & Virtual Azure & A1 Basic or equivalent to AWS Free Tier\\
\hline
% \hline this adds another line on the bottom
\end{tabular}
\newline
\newline

\begin{itemize}
\item May need to download from Azure and convert.	
\item Can download the VHD and convert.
\end{itemize}
%-------------------------------------------------------------------------------
% Write a draft hypothesis
%-------------------------------------------------------------------------------
\newpage
\section*{Write a draft hypothesis}
\addcontentsline{toc}{section}{Write a draft hypothesis}

Write a draft hypothesis
\newline
The large scale deployment of cloud resources onto low cost commodity hardware, in heavily deployed 
Software Defined Networks 
\newline
Next line adds latin
Line is rem'd out with a percentage sign
\newline
%\lipsum[2-2]

%-------------------------------------------------------------------------------
% Identify specific research questions/ objectives
%-------------------------------------------------------------------------------
\newpage
\section*{Identify specific research questions/ objectives}
\addcontentsline{toc}{section}{Identify specific research questions/ objectives}
Identify specific research questions/ objectives
\newline
Identify specific research questions/ objectives
Attempt to firewalk a pfSense in the 5 declared scenarios and see if they all perform in the same way.
Then compare the results with VNS3 in at least 3 of the 5 situations
\newline
Next line adds latin
Line is rem'd out with a percentage sign
\newline
%\lipsum[2-2]

%-------------------------------------------------------------------------------
% Identify possible sources of data
%-------------------------------------------------------------------------------
\newpage
\section*{Identify possible sources of data}
\addcontentsline{toc}{section}{Identify possible sources of data}
Identify possible sources of data or how to gather data in order to develop a problem solution
\newline
Small test lab with a DC/ IIS Web Server/ File Server/ LAMP Stack/ INetSIM box.
\newline
\newline
\maketitle
% \begin{tabular}{|l|l|l|} Formats the Left, Centre or Right l,c,r
\begin{tabular}{|l|l|l|}
\hline
\textbf{Server} & \textbf{Service} & \textbf{Port}\\
\hline
DC & DNS & 53\\
\hline
DC & LDAP & 389\\
\hline
DC & Sysvol & TCP 139\\
\hline
DC & Sysvol & UDP 138\\
\hline
IIS & HTTP & 80\\
\hline
IIS & HTTPS & 443\\
\hline
File Server & SMB & TCP 135-139\\
\hline
File Server & SMB & UDP 135-139\\
\hline
File Server & Direct SMB & TCP/UDP 445\\
\hline
File Server & Direct SMB & TCP/UDP 445\\
\hline
Azure File Share & SMB 3 & TCP 445\\
\hline
% \hline this adds another line on the bottom
\end{tabular}
\newline
\newline
Next line adds Latin
\newline
\newline
% \lipsum[2-2]
%-------------------------------------------------------------------------------
% REFERENCES
%-------------------------------------------------------------------------------
\newpage
\section*{References}
\addcontentsline{toc}{section}{References}
\textbf{For Reference an online quote}
\newline
R\&D Systems, 2013. \textit{Technical Information. Ischemia/Reperfusion Injury.} [online] Available at: <\url{http://www.rndsystems.com/cb_detail_objectname_SP96_Ischemia.aspx}> [Accessed 28 October 2013].
\newline
\newline
\textbf{For Reference a book Quote}
\newline
Shah, S., Baliga, R., Rajapurkar, M. and Fonseca, V., 2007. Oxidants in Chronic Kidney Disease, \textit{Journal of the American Society of Nephrology,} 18(1), pp. 16-28.
\newline
\newline
Firewalking a Firewall, \textit{fzuckerman} [online] Available at: <\url{https://fzuckerman.wordpress.com/2016/10/12/firewall-exploits-firewalking/}> [Accessed 8 October 2018].
\newline
\newline
VNS3:vpn VPN Appliance,{Cohesive Networks} ,[online] Available at: <\url{https://cohesive.net/products/vns3vpn,}> [Accessed 8 October 2018].
\newline
\newline
VNS3:vpn Documentation , {Cohesive Networks} ,[online] Available at: <\url{https://cohesive.net/support/product-resources,}> [Accessed 8 October 2018].
\newline
\newline
pfSense Download, \textit{https://www.pfsense.org/download/,} Download page, current version.
\newline
\newline
pfSense Documentation, \textit{https://www.netgate.com/docs/pfsense/,} Documentation Page , Current self service support.
\newline
\newline
Using Latex
https://en.wikipedia.org/wiki/LaTeX
\newline
\newline
\maketitle
  \LaTeX{} is a document preparation system for
  the \TeX{} typesetting program.
Latex Tables
\newline
<\url{https://www.overleaf.com/latex/examples/tables/jmcytxzgxjfk}>
\newline
Indenting a whole paragraph
\newline
<\url{https://tex.stackexchange.com/questions/35933/indenting-a-whole-paragraph}>

\end{document}

%-------------------------------------------------------------------------------
% SNIPPETS
%-------------------------------------------------------------------------------

%\begin{figure}[!ht]
%	\centering
%	\includegraphics[width=0.8\textwidth]{file_name}
%	\caption{}
%	\centering
%	\label{label:file_name}
%\end{figure}

%\begin{figure}[!ht]
%	\centering
%	\includegraphics[width=0.8\textwidth]{graph}
%	\caption{Blood pressure ranges and associated level of hypertension (American Heart Association, 2013).}
%	\centering
%	\label{label:graph}
%\end{figure}

%\begin{wrapfigure}{r}{0.30\textwidth}
%	\vspace{-40pt}
%	\begin{center}
%		\includegraphics[width=0.29\textwidth]{file_name}
%	\end{center}
%	\vspace{-20pt}
%	\caption{}
%	\label{label:file_name}
%\end{wrapfigure}

%\begin{wrapfigure}{r}{0.45\textwidth}
%	\begin{center}
%		\includegraphics[width=0.29\textwidth]{manometer}
%	\end{center}
%	\caption{Aneroid sphygmomanometer with stethoscope (Medicalexpo, 2012).}
%	\label{label:manometer}
%\end{wrapfigure}

%\begin{table}[!ht]\footnotesize
%	\centering
%	\begin{tabular}{cccccc}
%	\toprule
%	\multicolumn{2}{c} {Pearson's correlation test} & \multicolumn{4}{c} {Independent t-test} \\
%	\midrule	
%	\multicolumn{2}{c} {Gender} & \multicolumn{2}{c} {Activity level} & \multicolumn{2}{c} {Gender} \\
%	\midrule
%	Males & Females & 1st level & 6th level & Males & Females \\
%	\midrule
%	\multicolumn{2}{c} {BMI vs. SP} & \multicolumn{2}{c} {Systolic pressure} & \multicolumn{2}{c} {Systolic Pressure} \\
%	\multicolumn{2}{c} {BMI vs. DP} & \multicolumn{2}{c} {Diastolic pressure} & \multicolumn{2}{c} {Diastolic pressure} \\
%	\multicolumn{2}{c} {BMI vs. MAP} & \multicolumn{2}{c} {MAP} & \multicolumn{2}{c} {MAP} \\
%	\multicolumn{2}{c} {W:H ratio vs. SP} & \multicolumn{2}{c} {BMI} & \multicolumn{2}{c} {BMI} \\
%	\multicolumn{2}{c} {W:H ratio vs. DP} & \multicolumn{2}{c} {W:H ratio} & \multicolumn{2}{c} {W:H ratio} \\
%	\multicolumn{2}{c} {W:H ratio vs. MAP} & \multicolumn{2}{c} {\% Body fat} & \multicolumn{2}{c} {\% Body fat} \\
%	\multicolumn{2}{c} {} & \multicolumn{2}{c} {Height} & \multicolumn{2}{c} {Height} \\
%	\multicolumn{2}{c} {} & \multicolumn{2}{c} {Weight} & \multicolumn{2}{c} {Weight} \\
%	\multicolumn{2}{c} {} & \multicolumn{2}{c} {Heart rate} & \multicolumn{2}{c} {Heart rate} \\
%	\bottomrule
%	\end{tabular}
%	\caption{Parameters that were analysed and related statistical test performed for current study. BMI - body mass index; SP - systolic pressure; DP - diastolic pressure; MAP - mean arterial pressure; W:H ratio - waist to hip ratio.}
%	\label{label:tests}
%\end{table}
